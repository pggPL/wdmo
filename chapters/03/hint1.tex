\hints{Bijekcje i bajki kombinatoryczne}

\begin{hints_list}
	\item Na ile sposobów spośród $n$ osób możesz wybrać drużynę i mianować jednego z jej członków kapitanem?
	\item Suma liczb rozpatrywanego zbioru wynosi $55$.
	\item Podzielmy $2n$ osób na dwie grupy po $n$ osób. Załóżmy, że z pierwszej grupy wybieramy $k$ osób. Na ile sposobów możesz to zrobić?
	\item ,,Jeśli pewna permutacja należy do $A_1$, to jeśli pomnożymy wszystkie jej elementy przez $2$, to będzie należała do $A_2$.'' To stwierdzenie nie jest poprawne, ale wyraża pomysł na to zadanie.
	\item Rozpatrz liczbę podziałów $kn$ osób na $k$ grup po $n$ osób. Nie bierz pod uwagę żadnej kolejności grup, ani kolejności osób w grupie. 
	\item Zbiory, których średnia arytmetyczna jest liczbą całkowitą, zawierające więcej niż $1$ element podziel na pary.
	\item Wykaż, że obie strony równości to liczba słów, które składają się z $n + 1$ liter $A$ oraz $r$ liter $B$.
\end{hints_list}