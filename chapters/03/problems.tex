\begin{problem}{1} 
	Dane są liczby całkowite $n$ i $k$. Wykaż, że
	\[
		\sum^{n}_{k=0} k \cdot {{n}\choose{k}} = n \cdot 2^{n - 1}.
	\]
\end{problem}

\begin{problem}{2}
	Wyznacz liczbę podzbiorów zbioru $\{1, 2, 3, ..., 10\}$, których suma wynosi co najmniej $28$.
\end{problem}

\begin{problem}{3} 
	Udowodnić, że dla wszystkich dodatnich liczb całkowitych $n$ zachodzi równość
	\[
	    \sum^{n}_{k=0} {{n}\choose{k}}^2 = {{2n}\choose{n}}.
	\]
\end{problem}

\begin{problem}{4}
	Dana jest liczba pierwsza $p \geqslant 3$. Niech $A_k$ oznacza zbiór permutacji $(a_1, a_2, ..., a_p)$ zbioru $\{1, 2, 3,..., p\}$, dla których liczba
	\[
		a_1 + 2a_2 + 3a_3 + ... + pa_p - k
	\]
	jest podzielna przez $p$. Wykazać, że zbiory $A_1$ oraz $A_2$ mają tyle samo elementów.
\end{problem}

\begin{problem}{5}
	Wykaż, że dla dowolnych dodatnich liczb całkowitych $n$, $k$ liczba	$(kn)!$
	jest podzielna przez liczbę $(n!)^k \cdot k!$.
\end{problem}


\begin{problem}{6}
	Dana jest liczba całkowita $n$. Niech $T_n$ oznacza liczbę takich podzbiorów zbioru $\{1, 2, 3, ..., n\}$, że ich średnia arytmetyczna jest liczbą całkowitą. Wykazać, że liczba $T_n - n$ jest parzysta.
\end{problem}


\begin{problem}{7}
	Niech $n$, $k$, $r$ będą dodatnimi liczbami całkowitymi. Wykaż, że
	\[
		\sum^{r}_{k=0} {{n + k}\choose{k}} = {{n + r + 1}\choose{r}} .
	\]
\end{problem}

