\begin{problem}{1}
	Dana jest liczba pierwsza $p>3$. Wykazać, że istnieje taki ciąg arytmetyczny dodatnich liczb całkowitych $\{x_i\}$, że $x_1x_2\cdot ... \cdot x_p$ jest kwadratem liczby całkowitej.
\end{problem}


\begin{problem}{2}
	Dany jest ciąg $a_1, a_2, ..., a_{2n + 1}$ liczb rzeczywistych, że dla dowolnych liczb całkowitych ${2n + 1 \geqslant i, j \geqslant 1}$ zachodzi równość
	\[
		a_i + a_j \geqslant |i - j|.
	\]
	Wyznaczyć najmniejszą możliwą sumę elementów tego ciągu.
\end{problem}

\begin{problem}{3}
	Ciąg ${u_n}$ dany jest wzorem $u_0=1, u_{n+1}=\frac{u_n}{u_n+3}$. Wykaż, że $ u_1+u_2+...+u_{2021}<1$
\end{problem}


\begin{problem}{4}
	Znajdź wszystkie liczby rzeczywiste $x$, że ciąg dany wzorem
	\[
		a_0 = x, \quad a_n = 2^n - 3a_n,
	\]
	dla dowolnej liczby naturalnej n spełnia zależność
	\[
		a_{n} > a_{n - 1}.
	\]
\end{problem}


\begin{problem}{5}
	Dany jest ciąg
	\[
		a_0 = 1, \quad a_{n + 1} = a_n + \frac{1}{a_n}.
	\]
	Rozstrzygnąć, czy $a_{5000} > 100$.
\end{problem}

\begin{problem}{6}
	Wyznaczyć wszystkie dodatnie liczby rzeczywiste $\alpha$, dla których istnieje ciąg dodatnich liczb rzeczywistych $x_1$, $x_2$, $x_3$, $\dots$ o tej własności, że dla wszystkich $n\geqslant 1$:
	\[
		x_{n+2} = \sqrt{\alpha x_{n+1} - x_n}.
	\]
\end{problem}

\begin{problem}{7}
	Dany jest ciąg
	\[
		a_0 = 6, \quad a_n = a_{n - 1} + NWD(n, a_{n - 1}).
	\]
	Wykazać, że $NWD(n, a_{n - 1})$ jest liczbą pierwszą lub jest równe 1.
\end{problem}


