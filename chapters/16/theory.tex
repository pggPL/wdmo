% Rozdział 16 – ciągi

% skrypt bardziej zadaniowy

\theory{Ciągi}

% wyprowadzenie wzoru na ciąg Fibonacciego
% rozbite na lematy

\heading{Ciąg Fibonacciego}

% definicja tego ciągu

\heading{Wzór ogólny na wyrazy ciągu Fibonacciego}



\heading{Przykład 1}

\noindent
Wykazać, że dla każdej liczby całkowitej $M > 1$ istnieje taka dodatnia liczba całkowita $n$, że $F_n$ jest podzielne przez $M$.

\heading{Rozwiązanie}


% IMO 2018 P2
\heading{Przykład 2}

\noindent
Znajdź wszystkie liczby całkowite $n \geq 3$, dla których istnieją liczby rzeczywiste $a_1, a_2, \dots a_{n + 2}$ spełniające $a_{n + 1} = a_1$, $a_{n + 2} = a_2$ oraz
\[
	a_ia_{i + 1} + 1 = a_{i + 2},
\]
dla $i = 1, 2, \dots, n$.

\heading{Rozwiązanie}

% wspomnieć o rozpatrzaniu małych przypadków

% powiedzieć, że czasem trzeba robić jakieś przekształcenia znikąd, bawić się
% nie wszystko da się zawsze wyprowadzić wprost