\begin{problem}{1} 
	Wykazać, że dla dowolnej liczby rzeczywistej $x$ zachodzi
	\[
		x^2 + \frac{1}{x} \geqslant \frac{3}{2}\sqrt[3]{2}.
	\]
\end{problem}

\begin{problem}{2} 
	Dane są takie dodatnie liczby rzeczywiste $a_1, \;a_2,\; a_3,\; ...,\; a_n$, że $a_1a_2a_3...a_n = 1$. Wykazać, że
	\[
		(a_1 + a_2)(a_2 + a_3)\cdot ... \cdot (a_{n-1} + a_n)(a_n + a_1) \geqslant 2^n.
	\]
\end{problem}

\begin{problem}{3} 
	Udowodnić, że dla dowolnych dodatnich liczb rzeczywistych zachodzi nierówność
	\[
		\frac{a}{b + c} + \frac{b}{a + c} + \frac{c}{a + b} \geqslant \frac{3}{2}.
	\]
\end{problem}

\begin{problem}{4} 
	Dane są takie dodatnie liczby rzeczywiste $a$, $b$, $c$, że $abc = 1$. Wykazać, że
	\[
		a^2 + b^2 + c^2 \geqslant a + b + c.
	\]
\end{problem}

\begin{problem}{5} 
	Udowodnić, że dla dowolnych liczb rzeczywistych $a$, $b$, $c$ zachodzi nierówność
	\[
		\frac{a^2}{a + b} + \frac{b^2}{b + c} \geqslant \frac{3a + 2b - c}{4}. 
	\]
\end{problem}