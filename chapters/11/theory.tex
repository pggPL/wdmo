% Rozdział 11 – gry

\theory{Gry}

% coś o tym, czym jest gra i strategia wygrywająca

\heading{Przykład 1}

Ania i Bartek grają w następującą grę. Mają do dyspozycji stół w kształcie koła oraz dowolnie wiele monet o jednakowej średnicy, mniejszej od średnicy stołu. W każdym ruchu gracz kładzie monetę na stole tak, aby nie przykrywała żadnej innej monety. Gdy wykonanie ruchu nie jest możliwe, gracz którego jest kolej przegrywa. Rozstrzygnąc który z graczy ma startegię wygrywającą.

\heading{Rozwiązanie}

% komentarz – warto poszukać symetrii, zrobić małe przypadki



% analiza stanów – przykład gry z kamieniami
\heading{Przykład 2}

Na stole leży $10000$ kamieni. Ruch polega na zabraniu ze stołu $2$ lub $5$ kamieni. Dwaj gracze wykonują ruch na przemian, gdy gracz nie może wykonać ruchu przegrywa. Rozstrzygnąć, który z graczy – pierwszy czy drugi – ma strategię wygrywającą.

\heading{Rozwiązanie}

% tabelka ze stanami

% komentarz, że to podejście jest bardziej ogólne

