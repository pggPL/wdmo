\begin{problem}{1}
	Na tablicy napisano liczbę całkowitą dodatnią. W każdym ruchu zmazujemy napisaną liczbę $n$ na tablicy i piszemy nową liczbę. Jeśli $n$ była parzysta to piszemy na tablicy liczbę $\frac{1}{2}n$. Jeśli zaś liczba $n$ była nieparzysta, to zapisujemy jedną z liczb $3n - 1$ lub $3n + 1$. Czy -- niezależnie od tego, jaką liczbę zapisano na początku -- możemy, po skończenie wielu krokach, uzyskać na tablicy jedynkę?
\end{problem}


\begin{problem}{2}
	W lewym dolnym rogu planszy $m \times n$ stoi pionek. W każdym ruchu może zostać on przesunięty o dowolną liczbę pól w górę lub o dowolną liczbę pól w prawo. Wygrywa gracz, który postawi figurę w prawym górnym rogu. Rozstrzygnąć dla jakich wartości $(m, n)$ pierwszy gracz ma strategię wygrywającą.
\end{problem}

\begin{problem}{3}
	Na tablicy zapisano liczbę $10000000$. W każdym ruchu, o ile przed nim była zapisana liczba $n$, gracz zastępuje ją liczbą $n - 1$ lub $\left\lfloor\frac{n + 1}{2}\right\rfloor$. Gracz, który zapisze liczbę $1$ wygrywa. Który z graczy – pierwszy czy drugi – ma strategię wygrywającą?
\end{problem}

\begin{problem}{4}
	Dwaj gracze na przemian stawiają kółko i krzyżyk w polach nieskończonej planszy. Gracz wygrywa, gdy istnieje kwadrat $2\times2$ ułożony z jego symboli. Wykazać, że drugi gracz może grać tak, aby pierwszy gracz nie był w stanie wygrać.
\end{problem}

\begin{problem}{5}
	Nauczyciel wraz z 30 uczniami gra w grę na nieskończonej kartce w kratkę. Zaczyna on, po czym ruch wykonuje każdy z 30 uczniów, po czym znów nauczyciel, po czym uczniowie i tak dalej. W każdym ruchu należy pokolorować jeden z boków kratki, który nie został wcześniej pokolorowany. Nauczyciel wygrywa, gdy na planszy znajduje się prostokąt $2 \times 1$ lub $1 \times 2$, że wszystkie jego boki są pokolorowane, ale odcinek wewnątrz niego nie jest. Udowodnij, że ma on strategię wygrywającą.
\end{problem}

\begin{problem}{6}
	$20$ dziewczyn usiadło w kółku. Na początku jedna z nich trzyma $N < 19$ kamieni. W każdym ruchu jedna z dziewczyn, która posiada co najmniej dwa kamienie daje po jednym każdej ze swoich sąsiadek. Gra kończy się, gdy każda z dziewczyn trzyma co najwyżej jeden kamień. Wykazać, że gra musi się skończyć po skończonej liczbie ruchów.
\end{problem}