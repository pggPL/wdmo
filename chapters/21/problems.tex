\begin{problem}{1}
	Danych jest $n$ punktów i $n$ prostych, przy czym żadne trzy punkty nie są współliniowe oraz żadne dwie proste nie są równolgełe. Wykazać, że można tak pogrupować punkty i proste w pary, aby odcinki łączące punkt z jego rzutem prostokątnym na przyporządkowaną do niego prostą, nie przecinały się.
\end{problem}

\begin{problem}{2}
	Niech $n \geqslant 2$ będzie dodatnią liczbą całkowitą i niech $S$ będzie zbiorem zawierającym dokładnie $n$ różnych liczb rzeczywistych. Niech $T$ będzie zbiorem wszystkich liczb postaci $x_i + x_j$, gdzie $x_i$ oraz $x_j$ są różnymi elementami $S$. Wykazać, że zbiór $T$ zawiera co najmniej $2n - 3$ elementów.
\end{problem}

\begin{problem}{3}
	Niech $p_1, p_2, p_3, \ldots$ będą kolejnymi liczbami pierwszymi, zaś $x_0$ niech będzie liczbą rzeczywista pomiędzy 0 i 1. Dla każdej dodatniej liczby całkowitej $k$, przyjmijmy
\[
	x_k = \begin{cases} 
	0 & \text{jeśli} \; x_{k-1} = 0, \\
	\left\{ \dfrac{p_k}{x_{k-1}} \right\} & \text{jeśli} \; x_{k-1} \neq 0, 
	\end{cases}  
\]
gdzie $\{x\}$ oznacza część ułamkową $x$. (Część ułamkowa $x$ jest równa $x - \lfloor x \rfloor$) Znajdź wszystkie liczby $x_0$ spełniające $0 < x_0 < 1$, dla których ciąg $x_0, x_1, x_2, \ldots$ od pewnego momentu jest równy 0.
\end{problem}

\begin{problem}{4}
	Dana jest plansza będąca kratką $1000\times 1000$. W każdą kratkę wpisano strzałkę, która wskazuje jeden z boków kratki. W jednej z tych kratek stoi mrówka. Co sekundę wykonuje ona ruch -- rusza się na pole wskazane przez strzałkę w polu, na którym stoi, a następnie obraca tę strzałkę o $90\degree$ zgodnie z ruchem wskazówek zegara. Wykazać, że mrówka wyjdzie kiedyś z wyjściowej planszy.
\end{problem}

\begin{problem}{5}
Na tablicy zapisano pewną liczbę naturalną. W każdym ruchu możemy zastąpić aktualnie zapisaną liczbę $x$ jedną z liczb 
\[
    2x + 1 \quad  \text{lub} \quad \dfrac{x}{x + 2}.
\]
Wykazać, że jeśli na tablicy pojawiła się liczba 2000, to była tam ona od samego początku.
\end{problem}


\begin{problem}{6}
Niech $n$ będzie dodatnią liczbą całkowitą. Syzyf wykonuje ruchu na planszy, która zawiera $n + 1$ pól, ponumerowanych $0$ do $n$ od lewej do prawej. Początkowo $n$ kamieni znajduje się nie polu o numerze $0$, zaś inne pola są puste. W każdym ruchu, Syzyf wybiera niepuste pole, niech ono zawiera $k$ kamieni, bierze jeden kamień i przesuwa go w prawo o co najwyżej $k$ pól (kamień musi pozostać na planszy). Celem Syzyfa jest przeniesienie wszystkich kamieni z pola o numerze $0$ na pole o numerze $n$.
Wykazać, że Syzyf nie może osiągnąć swojego celu w mniej niż
\[ 
	\left \lceil \frac{n}{1} \right \rceil + \left \lceil \frac{n}{2} \right \rceil + \left \lceil \frac{n}{3} \right \rceil + \dots + \left \lceil \frac{n}{n} \right \rceil 
\]
ruchach.
\end{problem}






