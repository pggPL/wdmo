%Source: https://docs.google.com/viewer?a=v&pid=sites&srcid=ZGVmYXVsdGRvbWFpbnxpbW9jYW5hZGF8Z3g6NDA2NGM2NjkwZTJiYTg2Ng Canada Training Camp
\begin{problem}{1}
	Udowodnić, że za pomocą monet dwuzłotowych i pięciozłotowych można zapłacić każdą kwotę większą niż $4$ złote, bez konieczności wydawania reszty.
\end{problem}

%Source: https://docs.google.com/viewer?a=v&pid=sites&srcid=ZGVmYXVsdGRvbWFpbnxpbW9jYW5hZGF8Z3g6NWVkYTk2MWY5ODMzZGMxNA
\begin{problem}{2}
	Dana jest liczba naturalna $n$. Niech $\mathcal{S}$ będzie zbiorem wszystkich punktów postaci $(a, b)$, gdzie $a$ i $b$ są liczbami ze zbioru $\{0, 1, 2, ..., n\}$. Wykazać, że jeśli pewien zbiór prostych zawiera każdy z tych punktów, to zawiera on co najmniej $n + 1$ prostych. 
\end{problem}

%Source: https://om.mimuw.edu.pl/static/app_main/camps/oboz2015.pdf P 9
\begin{problem}{3}
	Wykazać, że istnieje taka dodatnia liczba całkowita, że jest ona podzielna przez $2^{1000}$, oraz ma w zapisie dziesiętnym jedynie cyfry $1$ i $2$.
\end{problem}

%Source: https://om.mimuw.edu.pl/static/app_main/camps/oboz2018.pdf P 26
\begin{problem}{4}
	Niech $k$ będzie dodatnią liczbą całkowitą. Na przyjęciu spotkało sie $n \geqslant 2$ gości, spośród których niektórzy znają się. Okazało się, że dla każdego niepustego podzbioru gości $A$ istnieje osoba, która zna co najwyżej $k$ osób z $A$. Wykazać, że istnieje co najwyżej $2^k \cdot n$ klik.
\end{problem}

%Source: https://docs.google.com/viewer?a=v&pid=sites&srcid=ZGVmYXVsdGRvbWFpbnxpbW9jYW5hZGF8Z3g6NDA2NGM2NjkwZTJiYTg2Ng P 2
\begin{problem}{5}
	Znajdź wszystkie funkcje $f$ z dodatnich liczb całkowitych w dodatnie liczby całkowite, które dla każdej liczby dodatniej całkowitej $n$ spełniają nierówności
	\[
		(n - 1)^2 < f(n)f(f(n)) < n^2 + n.
	\]
\end{problem}

%Source: https://om.mimuw.edu.pl/static/app_main/camps/oboz2019.pdf P 14
\begin{problem}{6}
	Niech $\mathcal{R}$ będzie rodziną zbiorów $1000$-elementowych. Moc $\mathcal{R}$ wynosi co najmniej $1000 \cdot 999^{1000}$ Wykazać, że istnieje $1000$-elementowa rodzina $\mathcal{G}$, będąca podrodziną $\mathcal{R}$, oraz taki zbiór $X$, że dla dowolnych zbiorów $A, B \in \mathcal{R}$ zachodzi $A \cap B = X$.
\end{problem}

