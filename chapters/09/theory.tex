% Rozdział 9 – Indukcja matematyczna 2

\theory{Indukcja matematyczna 2}

% dopisać, że indukcji trzeba zawsze próbować

\noindent
Indukcja, o której pisaliśmy wcześniej, korzystała z faktu, że jeśli teza zachodzi dla pewnej liczby $k$, to zachodzi również dla liczby $k + 1$. W nastepującym przykładzie pokażemy nieco ogólniejszą metodę, zwaną indukcją zupełną.

\vspace{10px}

\heading{Przykład 1}

\noindent
Niech $F_0 = 0$, $F_1 = 1$, $F_{n + 1} = F_n + F_{n - 1}$ dla $n \geqslant 1$ będą kolejnymi liczbami Fibonacciego. Wykazać, że każda dodatnia liczba całkowita może być przedstawiona w postaci sumy parami różnych liczb Fibonacciego.

\vspace{10px}

\heading{Rozwiązanie}


\vspace{10px}

% https://docs.google.com/viewer?a=v&pid=sites&srcid=ZGVmYXVsdGRvbWFpbnxpbW9jYW5hZGF8Z3g6NDA2NGM2NjkwZTJiYTg2Ng
% Wspomnieć o umocnieniu
\heading{Przykład 2}

\noindent
Wykaż, że
\[
	\frac{1}{2} \cdot \frac{3}{4} \cdot ... \cdot \frac{2n - 1}{2n} < \frac{1}{\sqrt{3n}}.
\]