\newpage
\solutions{Równania funkcyjne i wielomianowe}

\begin{problem}{1}
	Dana jest funkcja spełniająca dla dowolnej liczby rzeczywistej $x$ równość
	\[
		f(f(x)) - f(x) = x.
	\]
	Znaleźć liczbę liczb rzeczywistych $a$, takich, że $f(f(a)) = 0$.
\end{problem}

\answer{Jest jedna taka liczba i jest ona równa 0.}

\noindent
Podstawmy $x = 0$, aby otrzymać
\[
	f(f(0)) = f(0).
\]
Wstawiając $x = f(0)$ otrzymujemy
\[
	f(0) = f(f(f(0))) - f(f(0)) = f(f(0)) - f(f(0)) = 0.
\]
Czyli $a = 0$ jest rozwiązaniem.

\vspace{10px}

\noindent
Załóżmy, że $f(f(a)) = 0$. Z wyjściowego równania, wstawiając $f(a)$ za $x$, wynika wówczas, że
\begin{align*}
	f(f(f(a))) - f(f(a)) &= f(a), \\
	f(0) - 0 &= f(a), \\
	0 = f(a).
\end{align*}
Podstawiając zaś $x = a$ do danego równania mamy
Stąd
\[
	0 = f(f(a)) - f(a) = a.
\]
Oznacza to, że żadna liczba różna od $0$ nie może być rozwiązaniem danego równania.

\begin{problem}{2}
	Znajdź wszystkie funkcje  $ f: \mathbb{Z}_{>0} \to \mathbb{Z}_{>0}$ takie, że  $f(f(n)) + (f(n))^2 = n^2 + 3n + 3$.
\end{problem}

\answer{Jedyną funkcją spełniająca warunki zadania jest $f(n) = n + 1$.}

\noindent
Zauważmy, że dla $n = 1$ mamy
\[
	f(f(1)) + f(1)^2 = 7.
\]
Jeśli $f(1) = 1$, to $f(f(1)) + f(1)^2 = 1 + 1 \neq 7$. Jeśli zaś $f(1) \geqslant 3$, to
\[
	f(f(1)) + f(1)^2 > 3^2 > 7.
\]
Stąd $f(1) = 2$. Wykażemy indukcyjnie, że $f(n) = n + 1$. Załóżmy, że ta równość zachodzi dla liczb nie większych niż $n$. Wówczas
\begin{align*}
	n^2 + 3n + 3 &= f(f(n)) + f(n)^2 = f(n + 1) + (n + 1)^2, \\
	f(n + 1) &= n + 2,
\end{align*}
co kończy dowód indukcyjny. 

\begin{problem}{3}
	Znajdź wszystkie funkcje $f:\mathbb{R} \to \mathbb{R}$, że dla dowolnych liczb rzeczywistych $x$, $y$ zachodzi równość:
	\[
		f(x^2 + y) = f(x^{27} + 2y) + f(x^4).
	\]
\end{problem}

\answer{
	Jedyną funkcją spełniającą warunki zadania jest $f(x) = 0$.
}

\noindent
Rozpatrzmy takie $y$, aby 
\begin{align*}
	x^2 + y &= x^{27} + 2y \\
	x^2 - x^{27} &= y.
\end{align*}
Więc dla $y = x^2 - x^{27}$ mamy
\[
	f(x^2 + y) = f(x^{27} + 2y).
\]
A więc na mocy wyjściowego równania
\[
	f(x^4) = 0.
\]
Stąd $f(x) = 0$ dla dowolnej liczby nieujemnej $x$. Mamy też
\[
	f(x^2 + y) = f(x^{27} + 2y) + f(x^4) =  f(x^{27} + 2y)
\]
dla wszystkich liczb rzeczywistych $x$, $y$. Wstawiając $y - x^2$ w miejsce $y$ mamy
\[
	f(y) = f(x^{27} - 2x^2 + 2y).
\]
Niech $y$ będzie dowolną liczbą ujemną. Biorąc odpowiednio duże $x$ otrzymamy, że $x^{27} - 2x^2 + 2y > 0$. Wówczas
\[
	f(y) = f(x^{27} - 2x^2 + 2y) = 0.
\]
Pozostaje zauważyć, że $f(x) = 0$ spełnia warunki zadania.

\begin{problem}{4}
	Niech $f(n)$ będzie funkcją z dodatnich liczb całkowitych w dodatnie liczby całkowite. Wiadomo, że dla każdej dodatniej liczby całkowitej $n$ liczba  $f(f(n))$ jest liczbą dodatnich dzielników $n$. Wykazać, że jeśli $p$ jest liczbą pierwszą, wówczas $f(p)$ również jest liczbą pierwszą.
\end{problem}

\noindent
Niech $d(n)$ oznacza liczbę dzielników $n$. Wówczas
\begin{align*}
	f(f(n)) &= d(n), \\
	f(f(f(n))) &= f(d(n)), \\
	d(f(n)) &= f(d(n)).
\end{align*}
Dla $n = 2$ mamy
\[
	f(2) = f(d(2)) = d(f(2)).
\]
Zauważmy, że jeśli $f(2) > 2$, to $f(2) - 1$ nie byłoby dzielnikiem $f(2)$, stad $f(2)$ musiało by mieć mniej niż $f(2)$ dzielników. Rozpatrzmy dwa przypadki
\begin{itemize}
	\item Jeśli $f(2) = 1$, to gdy $p$ jest liczbą pierwszą, to 
	\[
		d(f(p)) = f(d(p)) = f(2) = 1,
	\]
	czyli $f(p)$ ma jeden dzielnik, czyli $f(p) = 1$. Mamy wówczas
	\[
		1 = f(3) = f(d(p^2)) = f(f(f(p^2))) = d(f(p^2)).
	\]
	Skoro $f(p^2)$ ma jeden dzielnik, to również jest równe $1$. Czyli
	\[
		2 = d(p) = f(f(p)) = f(1) = f(f(p^2)) = d(p^2) = 3,
	\]
	co daje nam sprzeczność.
	\item Gdy zaś $f(2) = 2$, to jeśli $p$ jest liczbą pierwszą, to 
	\[
		d(f(p)) = f(d(p)) = f(2) = 2.
	\]
	Skoro $f(p)$ ma dwa dzielniki, to jest liczbą pierwszą, czego należało dowieść.
\end{itemize}

\begin{problem}{5}
	Znajdź wszystkie funkcje $f:\mathbb{R}\longrightarrow\mathbb{R}$, że dla dowolnych liczb rzeczywistych $x$, $y$ zachodzi równość:
	\[
		f(x^2 + f(y)) = y + f(x)^2.
	\]
\end{problem}

\answer{Jedyną funkcją spełniającą warunki zadania jest $f(x) = x$.}

\noindent
Podstawmy $x = 0$ 
\[
	f(f(y)) = y + f(0)^2.
\] 
Przyjmijmy $a = f(0)$ jest stałe. Wykażemy, że $f$ jest różnowartościowa. Jeżeli dla pewnych liczb rzeczywistych $x_1, x_2$ zachodzi $f(x_1) = f(x_2)$, to 
\[
	x_1 + a^2 = f(f(x_1)) = f(f(x_2)) = x_2 + a^2,
\]
skąd $x_1 = x_2$.
Wyrażenie $y + a^2$ może przyjąć dowolną rzeczywistą wartość. Skoro $f(f(y)) = y + a^2$, to $f$ przyjmuje wszystkie wartości rzeczywiste.
Wiemy zatem, że istnieje liczba rzeczywista $b$ taka, że $f(b) = 0$. Podstawiając $x = b$ i $f(y)$ w miejsce $y$ w początkowej równości dostajemy 
\[
	f(b^2 + f(f(y))) = f(b^2 + y + a^2) = f(y),
\]
skąd, jako że $f$ to funkcja różnowartościowa, zachodzi 
\begin{align*}
	b^2 + y + a^2 = y, \\
	a^2 + b^2 = 0, \\
	a = b = 0.
\end{align*}
Wstawmy $x = 0$ do danego równania 
\[
	f(f(y)) = y.
\] 
Zaś podstawiając $y = f(y)$  otrzymamy 
\[
	f(x^2 + y) = f(y) + f(x)^2 \geqslant f(y).
\]
Skoro $x^2$ może być dowolną liczbą nieujemną, to z powyższej zależności wynika, że $f$ jest niemalejąca. Przyjmijmy, że dla pewnego rzeczywistego $x_0$ zachodzi $f(x_0) > x_0$. Wtedy, ponieważ $f$ jest niemalejąca,
mamy 
\[
	x_0 = f(f(x_0)) > f(x_0) \geqslant x_0,
\] 
co zajść nie może. Jeśli $f(x_0) < x_0$, to w analogiczny sposób otrzymujemy sprzeczność. Stąd $f(x_0) = x_0$ dla dowolnej liczby rzeczywistej $x_0$. Pozostaje zauważyć, że funkcja $f(x) = x$ spełnia warunki zadania.

\vspace{10px}

\begin{problem}{6}
	Wykazać, że istnieje taka funkcja $f:\mathbb{R}\longrightarrow\mathbb{R}$, że nie istnieje taka funkcja $g:\mathbb{R}\longrightarrow\mathbb{R}$, że zachodzi równość $f(x) = g(g(x))$.
\end{problem}

\noindent
Wykażemy, że taka funkcja istnieje. Rozpatrzmy funkcję $f$ daną jako
\[
	f(x) = 	\begin{cases} 
				x &\mbox{jeśli } x \notin \mathbb{Z}_{\geqslant 0} \\ 
				x + 1 & \mbox{jeśli } x \in \mathbb{Z}_{\geqslant 0}
			\end{cases}.
\]
Ta funkcja ma następujące cechy
\begin{itemize}
	\item jest różnowartościowa,
	\item przyjmuje wszystkie wartości rzeczywiste poza zerem.
\end{itemize}
Wykażemy, że $g$ jest różnowartościowa. Jeśli $g(a) = g(b)$, to
\[
	f(a) = g(g(a)) = g(g(b)) = f(b).
\]
Skoro zaś $f$ jest różnowartościowa, to mamy $a = b$.

\vspace{10px}

\noindent
Jeśli $g(x)$ przyjmuje wszystkie wartości rzeczywiste, to $g(g(x))$ również. Zaś funkcja $f(x)$ nie ma takiej własności -- sprzeczność.

\vspace{10px}

\noindent
Jeśli $g(x)$ ma co najmniej dwie ,,dziury'' -- tj. istnieją takie liczby $a$, $b$, które nie są przyjmowane przez $g$, to w szczególności nie są przyjmowane przez $g(g(x))$. Zaś funkcja $f$ ma jedną wartość, której nie przyjmuje. Toteż również w tym przypadku równość zajść nie może.

\vspace{10px}

\noindent
Załóżmy więc, że $g$ przyjmuje wszystkie wartości rzeczywiste poza pewną liczbą rzeczywistą $a$. Oczywiście $g(g(x))$ również nie przyjmie $a$. Skoro $f$ nie przyjmuje jedynie wartości zero, to $a = 0$. Zauważmy, że $g(g(x))$ nie przyjmuje wartości $g(0)$. Jeśli $g(g(t)) =  g(0)$ dla pewnego $t$, to na mocy różnowartościowości $g(t) = 0$. Jest to sprzeczność. Skoro $g(g(x))$ nie przyjmuje $g(0)$, a $f(x)$ nie przyjmuje jedynie $0$, to jeśli te funkcje mają być sobie równe, to $g(0) = 0$. Ale wówczas
\[
	1 = f(0) = g(g(0)) = g(0) = 0,
\]
co kończy dowód.